%File: anonymous-submission-latex-2023.tex
\documentclass[letterpaper]{article} % DO NOT CHANGE THIS
\usepackage[submission]{aaai23}  % DO NOT CHANGE THIS
\usepackage{times}  % DO NOT CHANGE THIS
\usepackage{helvet}  % DO NOT CHANGE THIS
\usepackage{courier}  % DO NOT CHANGE THIS
\usepackage[hyphens]{url}  % DO NOT CHANGE THIS
\usepackage{graphicx} % DO NOT CHANGE THIS
\urlstyle{rm} % DO NOT CHANGE THIS
\def\UrlFont{\rm}  % DO NOT CHANGE THIS
\usepackage{natbib}  % DO NOT CHANGE THIS AND DO NOT ADD ANY OPTIONS TO IT
\usepackage{caption} % DO NOT CHANGE THIS AND DO NOT ADD ANY OPTIONS TO IT
\frenchspacing  % DO NOT CHANGE THIS
\setlength{\pdfpagewidth}{8.5in} % DO NOT CHANGE THIS
\setlength{\pdfpageheight}{11in} % DO NOT CHANGE THIS
\usepackage{amsmath, amssymb, amscd, amsthm, amsfonts}
%
% These are recommended to typeset algorithms but not required. See the subsubsection on algorithms. Remove them if you don't have algorithms in your paper.
\usepackage{algorithm}
\usepackage{algorithmic}

%
% These are are recommended to typeset listings but not required. See the subsubsection on listing. Remove this block if you don't have listings in your paper.
\usepackage{newfloat}
\usepackage{listings}
\DeclareCaptionStyle{ruled}{labelfont=normalfont,labelsep=colon,strut=off} % DO NOT CHANGE THIS
\lstset{%
	basicstyle={\footnotesize\ttfamily},% footnotesize acceptable for monospace
	numbers=left,numberstyle=\footnotesize,xleftmargin=2em,% show line numbers, remove this entire line if you don't want the numbers.
	aboveskip=0pt,belowskip=0pt,%
	showstringspaces=false,tabsize=2,breaklines=true}
\floatstyle{ruled}
\newfloat{listing}{tb}{lst}{}
\floatname{listing}{Listing}
%
% Keep the \pdfinfo as shown here. There's no need
% for you to add the /Title and /Author tags.
\pdfinfo{
/TemplateVersion (2023.1)
}

% DISALLOWED PACKAGES
% \usepackage{authblk} -- This package is specifically forbidden
% \usepackage{balance} -- This package is specifically forbidden
% \usepackage{color (if used in text)
% \usepackage{CJK} -- This package is specifically forbidden
% \usepackage{float} -- This package is specifically forbidden
% \usepackage{flushend} -- This package is specifically forbidden
% \usepackage{fontenc} -- This package is specifically forbidden
% \usepackage{fullpage} -- This package is specifically forbidden
% \usepackage{geometry} -- This package is specifically forbidden
% \usepackage{grffile} -- This package is specifically forbidden
% \usepackage{hyperref} -- This package is specifically forbidden
% \usepackage{navigator} -- This package is specifically forbidden
% (or any other package that embeds links such as navigator or hyperref)
% \indentfirst} -- This package is specifically forbidden
% \layout} -- This package is specifically forbidden
% \multicol} -- This package is specifically forbidden
% \nameref} -- This package is specifically forbidden
% \usepackage{savetrees} -- This package is specifically forbidden
% \usepackage{setspace} -- This package is specifically forbidden
% \usepackage{stfloats} -- This package is specifically forbidden
% \usepackage{tabu} -- This package is specifically forbidden
% \usepackage{titlesec} -- This package is specifically forbidden
% \usepackage{tocbibind} -- This package is specifically forbidden
% \usepackage{ulem} -- This package is specifically forbidden
% \usepackage{wrapfig} -- This package is specifically forbidden
% DISALLOWED COMMANDS
% \nocopyright -- Your paper will not be published if you use this command
% \addtolength -- This command may not be used
% \balance -- This command may not be used
% \baselinestretch -- Your paper will not be published if you use this command
% \clearpage -- No page breaks of any kind may be used for the final version of your paper
% \columnsep -- This command may not be used
% \newpage -- No page breaks of any kind may be used for the final version of your paper
% \pagebreak -- No page breaks of any kind may be used for the final version of your paperr
% \pagestyle -- This command may not be used
% \tiny -- This is not an acceptable font size.
% \vspace{- -- No negative value may be used in proximity of a caption, figure, table, section, subsection, subsubsection, or reference
% \vskip{- -- No negative value may be used to alter spacing above or below a caption, figure, table, section, subsection, subsubsection, or reference

\setcounter{secnumdepth}{0} %May be changed to 1 or 2 if section numbers are desired.

% The file aaai23.sty is the style file for AAAI Press
% proceedings, working notes, and technical reports.
%

% Title

% Your title must be in mixed case, not sentence case.
% That means all verbs (including short verbs like be, is, using,and go),
% nouns, adverbs, adjectives should be capitalized, including both words in hyphenated terms, while
% articles, conjunctions, and prepositions are lower case unless they
% directly follow a colon or long dash
\title{An Adaptive Incremental Gradient Method With Support for Non-Euclidean Norms}

\begin{document}

\maketitle

\begin{abstract}
Stochastic variance reduced methods have shown strong performance in solving finite-sum problems.
%
However, these methods usually require the users to manually tune the step-size, which is time-consuming or even infeasible for some large-scale optimization tasks.
%
To overcome the problem, we propose and analyze several novel adaptive variants of the popular SAGA algorithm.
%
Eventually, we design a variant of Barzilai-Borwein step-size which is tailored for the incremental gradient method to ensure memory efficiency and fast convergence. 
%
We establish its convergence guarantees under general settings that allow non-Euclidean norms in the definition of smoothness and the composite objectives, which cover a broad range of applications in machine learning.
%
We improve the analysis of SAGA to support non-Euclidean norms, which fills the void of existing work.
%
Numerical experiments on standard datasets demonstrate a competitive performance of the proposed algorithm compared with existing variance-reduced methods and their adaptive variants.
\end{abstract}

\section{Introduction}
\label{s:section}
Many machine learning tasks involve solving the following optimization problem with a finite-sum structure:
\begin{equation}\label{problem1}
    \min_{x \in \mathbb{R}^d} \ f(x): = \frac{1}{n}\sum_{i \in [n]}f_i(x),
\end{equation}
where $x\in \mathbb{R}^d$ is the model parameter, each $f_i: \mathbb{R}^d \rightarrow \mathbb{R}$ is a loss function for the $i$-th training sample and $[n]:= \{1,\dots,n\}$.
%
More generally, the ``composite'' (or proximal) case covers a larger range of applications in machine
learning:
\begin{equation}\label{Problem}
    \min_{x \in \mathbb{R}^d} \ F(x) : = f(x) + h(x),
\end{equation}
where $h(\cdot): \mathbb{R}^d \rightarrow \mathbb{R}$ is a simple
and convex (but possibly non-differentiable) function, and the proximal operation of $h(\cdot)$ is easy to compute. %— few adaptive gradient methods are applicable in this setting.
%
Here, we also define $F_i(x) = f_i(x) + h(x)$ and $\nabla F_i(x) = \nabla f_i(x)+ \partial h(x)$ where $\partial h(x)$ denotes a sub-gradient of $h(\cdot)$ at $x.$
%
Throughout the paper, we denote $x^*$ as an optimal solution of Problem (\ref{Problem}).

However, when $n$ is very large, which is common in the modern learning tasks, gradient descent (GD) has a prohibitively high per-iteration cost.
%
%However, when $n$ is very large, GD has a prohibitively high per-iteration cost.
%
To mitigate the issue, stochastic gradient descent (SGD) \citep{10.1214/aoms/1177729586, DBLP:journals/siamjo/NemirovskiJLS09}, as an alternative, has been widely adopted to solve Problem (\ref{Problem}) in modern machine learning tasks.
%
Its simplest update process can be written as:
\begin{equation*}
    x_{k+1} = \arg \min_x \left\{\left\langle \widetilde{\nabla}_k, x \right\rangle  + \frac{1}{2\eta}\|x-x_k\|^2+ h(x)\right\},
\end{equation*}
where $\eta$ is the step-size, and the gradient estimator $\widetilde{\nabla}_k$ satisfies $\mathbb{E}[\widetilde{\nabla}_k] = \nabla f(x_k), \forall x \in \mathbb{R}^d$ to ensure the convergence.
%
The choice in SGD is to set $\widetilde{\nabla}_k=\nabla f_i(x)$.
%
However, due to the undiminishing variance $\mathbb{E}[\|\nabla f_i(x_k) - \nabla f(x_k) \|^2]$, SGD only converges at a sub-linear rate even if $F(\cdot)$ is strongly convex and smooth.

\paragraph{Variance reduction.}
%
Recently, some stochastic variance reduced (VR) methods have been proposed to solve the variance issue of SGD, such as SAG \citep{DBLP:conf/nips/RouxSB12}, SAGA \citep{DBLP:conf/nips/DefazioBL14}, SVRG \citep{DBLP:conf/nips/Johnson013}, SARAH \citep{DBLP:conf/icml/NguyenLST17} and SVRG-Loopless \citep{DBLP:conf/alt/KovalevHR20}.
%
There are also several accelerated or generalized variants of these methods proposed these years, to list a few, \cite{DBLP:conf/nips/Defazio16,DBLP:journals/mp/GowerRB21,DBLP:journals/corr/abs-2006-11573,DBLP:conf/nips/ZhouSC20}.
%
The methods use better choices of the gradient estimator $\widetilde{\nabla}_k$ with its variance reducing as the algorithm converges.
%
Based on this property, in theory, VR methods only need $\mathcal{O}((n+\kappa)\log(\frac{1}{\epsilon}))$ stochastic gradient evaluations to satisfy $\mathbb{E}[\|\nabla f(x) \|^2] \leq \epsilon$ or $\mathbb{E}[f(x)] - f(x^*) \leq \epsilon$, while GD typically needs $\mathcal{O}(n\kappa \log(\frac{1}{\epsilon}))$ evaluations and SGD requires $\mathcal{O}(\frac{1}{\mu\epsilon})$ complexity.
%

Consequently, VR methods are commonly used in practice, especially for convex problems such as logistic regression \citep{DBLP:journals/corr/abs-2102-09645}.
%
However, all the VR methods above require a constant step-size, which depends on the characteristics of Problem (\ref{Problem}), such as the smoothness constant.
%
In practice, these constants are often difficult to estimate.
%
In general, researchers adopt a computationally expensive grid-search method to find a step-size.
%
%Besides such a computational cost, a 
Yet such constant step-sizes might lead to poor empirical performance.

\paragraph{Adaptive VR methods.}
Consequently, researchers have also proposed some adaptive VR methods to address the issue in recent years.
%
\citet{DBLP:conf/nips/TanMDQ16} proposed SVRG-BB based on Barzilai-Borwein (BB) step-size \citep{barzilai1988two} and derived SGD-BB and SAG-BB as heuristics.
%
\citet{c:21} developed SVRG-AS and SARAH-AS.
%
Although the methods mentioned above are capable of automatically changing step-size, they still introduce additional hard-to-tune hyperparameters.
%
\citet{DBLP:conf/icml/LiWG20} designed another variants of SVRG-BB and SARAH-BB given the knowledge of some problem-dependent constants.
%
More recently, \citet{DBLP:journals/corr/abs-2102-09645} proposed AdaSVRG for convex objectives, which extends the adaptive step-size in AdaGrad to SVRG's inner loop.
%
However, the theoretical analysis of AdaSVRG requires a bounded domain, and thus the algorithm needs to perform projection in each iteration, which is rarely the case in machine learning tasks.
%
\citet{DBLP:journals/corr/abs-2102-09700} emphasized the importance of adapting to the local geometry and proposed an implicit adaptive strategy for SARAH (AI-SARAH).
%
However, AI-SARAH requires solving a sub-problem in each iteration, which could be expensive for general objective functions other than linear regressions.

In addition, most of the adaptive VR methods above are limited to solve Problem (\ref{problem1}) either in the convex setting or in the strongly convex setting.
%
That is, AdaSVRG is only analyzed in the convex case, and other methods are analyzed in the strongly convex case.
%
To the best of our knowledge, all the known adaptive VR methods and variants of SAGA do not work with a non-Euclidean norm in the definition of smoothness.
%
However, allowing arbitrary norms contribute to wide applications in many areas of computer science \citep{allen2014linear}.

Although a considerable amount of work has been done for adaptive SVRG and SARAH, another stochastic variance reduction method, SAGA, does not have any adaptive variant. 
%
SAGA, which is free of inner loop length tuning and usually has a better practical performance compared with SVRG, has been implemented in the scikit-learn package \citep{scikit-learn} (while SVRG and SARAH have not been) to solve linear regression tasks.
%
Therefore, an adaptive variant of SAGA is of great interest.

\bibliography{references} 

\bigskip

\end{document}
